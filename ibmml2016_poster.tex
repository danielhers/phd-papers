\documentclass[a0,portrait]{a0poster}

\usepackage{multicol} % This is so we can have multiple columns of text side-by-side
\columnsep=100pt % This is the amount of white space between the columns in the poster
\columnseprule=3pt % This is the thickness of the black line between the columns in the poster

\usepackage[svgnames]{xcolor} % Specify colors by their 'svgnames', for a full list of all colors available see here: http://www.latextemplates.com/svgnames-colors

\usepackage{times} % Use the times font
%\usepackage{palatino} % Uncomment to use the Palatino font

\usepackage{graphicx} % Required for including images
\graphicspath{{figures/}} % Location of the graphics files
\usepackage{booktabs} % Top and bottom rules for table
\usepackage[font=small,labelfont=bf]{caption} % Required for specifying captions to tables and figures
\usepackage{amsfonts, amsmath, amsthm, amssymb} % For math fonts, symbols and environments
\usepackage{wrapfig} % Allows wrapping text around tables and figures
\usepackage{tikz}
\usepackage{tikz-dependency}
\usetikzlibrary{arrows.meta}

\begin{document}

%----------------------------------------------------------------------------------------
%	POSTER HEADER 
%----------------------------------------------------------------------------------------

% The header is divided into two boxes:
% The first is 75% wide and houses the title, subtitle, names, university/organization and contact information
% The second is 25% wide and houses a logo for your university/organization or a photo of you
% The widths of these boxes can be easily edited to accommodate your content as you see fit

\begin{minipage}[b]{0.75\linewidth}
\veryHuge \color{NavyBlue} \textbf{Broad-Coverage Transition-Based UCCA Parsing}
\color{Black}\\ % Title
%\Huge\textit{An Exploration of Complexity}\\[2cm] % Subtitle
\huge \textbf{Daniel Hershcovich$^{1,2}$ \And Omri Abend$^2$ \And Ari Rappoport$^2$} \\[0.5cm] % Author(s)
\huge $^1$Edmond and Lily Safra Center for Brain Sciences \\
  $^2$School of Computer Science and Engineering \\
  Hebrew University of Jerusalem \\[0.4cm] % University/organization
\Large \texttt{\{danielh,oabend,arir\}@cs.huji.ac.il} \\
\end{minipage}
%
\begin{minipage}[b]{0.25\linewidth}
\includegraphics[width=0.49\linewidth]{huji_logo.png}
\includegraphics[width=0.49\linewidth]{elsc_logo.png}
\end{minipage}

\vspace{1cm} % A bit of extra whitespace between the header and poster content

%----------------------------------------------------------------------------------------

\begin{multicols}{2} % This is how many columns your poster will be broken into, a portrait poster is generally split into 2 columns

%----------------------------------------------------------------------------------------
%	ABSTRACT
%----------------------------------------------------------------------------------------

\color{Navy} % Navy color for the abstract

\begin{abstract}

  We present the first parser for UCCA, a
  cross-linguistically applicable framework for semantic
  representation, that builds on extensive
  typological work, and supports rapid annotation.
  UCCA poses a challenge for existing parsing techniques,
  as it exhibits reentrancy (resulting in DAG structures),
  discontinuous structures and non-terminal nodes corresponding
  to complex semantic units. To our knowledge, the conjunction
  of these formal properties is not supported by any existing parser.
  Our transition-based parser, using novel transition set
  and features, has value not just for UCCA parsing:
  its ability to handle more general graph structures will inform
  the development of parsers for other semantic DAG structures, 
  and in languages which frequently use discontinuous structures.

\end{abstract}

%----------------------------------------------------------------------------------------
%	INTRODUCTION
%----------------------------------------------------------------------------------------

\color{Black} % SaddleBrown color for the introduction

\section*{Introduction}

Universal Conceptual Cognitive Annotation \cite{abend2013universal},
is a cross-linguistically applicable semantic representation scheme,
building on the established Basic Linguistic Theory typological framework
\cite{Dixon:10b,Dixon:10a,Dixon:12}, and on Cognitive
Linguistics literature \cite{croft2004cognitive}.
UCCA has demonstrated applicability to multiple languages, including
English, French, German and Czech, support for rapid annotation,
and semantic stability in translation \cite{sulem2015conceptual}.
The scheme has proven useful for machine translation evaluation \cite{birch2016hume},
but its applicability has been so far limited by the absence of a UCCA parser,
a gap this paper addresses.

Formally, a UCCA structure is a DAG whose leaves correspond to the tokens of
the text. Nodes (or {\it units}) either correspond to a terminal or
to several sub-units (not necessarily contiguous) jointly viewed as a
single entity according to some semantic or cognitive consideration.
Edges bear a category, indicating the role of the sub-unit in the relation
that the parent represents.

\begin{center}
  \begin{tikzpicture}[level distance=4cm, sibling distance=4cm, -{Latex[length=5mm]},
      every circle node/.append style={fill=black}]
    \node (ROOT) [circle] {}
      child {node (After) {After} edge from parent node[left] {L}}
      child {node (graduation) [circle] {}
      {
        child {node {graduation} edge from parent node[left] {P}}
      } edge from parent node[left] {H} }
      child {node {,} edge from parent node[right] {U}}
      child {node (moved) [circle] {}
      {
        child {node (John) {John} edge from parent node[left] {A}}
        child {node {moved} edge from parent node[left] {P}}
        child {node [circle] {}
        {
          child {node {to} edge from parent node[left] {R}}
          child {node {Paris} edge from parent node[left] {C}}
        } edge from parent node[left] {A} }
      } edge from parent node[right] {H} }
      ;
    \draw[dashed,-{Latex[length=5mm]}] (graduation) to node [auto] {A} (John);
  \end{tikzpicture}
  \begin{tikzpicture}[level distance=5cm, sibling distance=4cm, -{Latex[length=5mm]},
      every node/.append style={midway},
      every circle node/.append style={fill=black}]
    \node (ROOT) [circle] {}
      child {node {John} edge from parent node[left] {A}}
      child {node [circle] {}
      {
      	child {node {gave} edge from parent node[left] {C}}
      	child {node (everything) {everything} edge from parent[white]}
      	child {node {up} edge from parent node[left] {C}}
      } edge from parent node[right] {P} }
      ;
    \draw[bend right,-{Latex[length=5mm]}] (ROOT) to[out=-20, in=180] node [left] {A} (everything);
  \end{tikzpicture}
  \begin{tikzpicture}[level distance=5cm, sibling distance=4cm, -{Latex[length=5mm]},
      every node/.append style={midway},
      every circle node/.append style={fill=black}]
    \node (ROOT) [circle] {}
      child {node [circle] {}
      {
        child {node {John} edge from parent node[left] {C}}
        child {node {and} edge from parent node[left] {N}}
        child {node {Mary} edge from parent node[left] {C}}
      } edge from parent node[left] {A} }
      child {node {went} edge from parent node[left] {P}}
      child {node {home} edge from parent node[left] {A}}
      ;
  \end{tikzpicture}
%    UCCA structures demonstrating three structural properties exhibited by
%    the scheme.
%    (\subref{fig:graduation}) includes a remote edge (dashed),
%    resulting in ``John'' having two parents.
%    (\subref{fig:gave}) includes a discontinuous unit (``gave ... up'').
%    (\subref{fig:home}) includes a coordination construction (``John and Mary'').
%    Legend: $P$ -- process (a scene's main relation), $A$ -- participant,
%    $L$ -- inter-scene linker, $H$ -- linked scene, $C$ -- center,
%    $R$ -- relator, $N$ -- connector, $U$ -- punctuation, $F$ -- function unit.
%    Pre-terminal nodes are omitted for brevity.
\end{center}

\color{DarkSlateGray} % DarkSlateGray color for the rest of the content

%----------------------------------------------------------------------------------------
%	MATERIALS AND METHODS
%----------------------------------------------------------------------------------------

\section*{Materials and Methods}

Fusce magna risus, molestie ut porttitor in, consectetur sed mi. Vestibulum ante ipsum primis in faucibus orci luctus et ultrices posuere cubilia Curae; Pellentesque consectetur blandit pellentesque. Sed odio justo, viverra nec porttitor vel, lacinia a nunc. Suspendisse pulvinar euismod arcu, sit amet accumsan enim fermentum quis. In id mauris ut dui feugiat egestas. Vestibulum ac turpis lacinia nisl commodo sagittis eget sit amet sapien.

%------------------------------------------------

\subsection*{Mathematical Section}

Nulla vel nisl sed mauris auctor mollis non sed. 

\begin{equation}
E = mc^{2}
\label{eqn:Einstein}
\end{equation}

Curabitur mi sem, pulvinar quis aliquam rutrum. (1) edf (2)
, $\Omega=[-1,1]^3$, maecenas leo est, ornare at. $z=-1$ edf $z=1$ sed interdum felis dapibus sem. $x$ set $y$ ytruem. 
Turpis $j$ amet accumsan enim $y$-lacina; 
ref $k$-viverra nec porttitor $x$-lacina. 

Vestibulum ac diam a odio tempus congue. Vivamus id enim nisi:

\begin{eqnarray}
\cos\bar{\phi}_k Q_{j,k+1,t} + Q_{j,k+1,x}+\frac{\sin^2\bar{\phi}_k}{T\cos\bar{\phi}_k} Q_{j,k+1} &=&\nonumber\\ 
-\cos\phi_k Q_{j,k,t} + Q_{j,k,x}-\frac{\sin^2\phi_k}{T\cos\phi_k} Q_{j,k}\label{edgek}
\end{eqnarray}
and
\begin{eqnarray}
\cos\bar{\phi}_j Q_{j+1,k,t} + Q_{j+1,k,y}+\frac{\sin^2\bar{\phi}_j}{T\cos\bar{\phi}_j} Q_{j+1,k}&=&\nonumber \\
-\cos\phi_j Q_{j,k,t} + Q_{j,k,y}-\frac{\sin^2\phi_j}{T\cos\phi_j} Q_{j,k}.\label{edgej}
\end{eqnarray} 

Nulla sed arcu arcu. Duis et ante gravida orci venenatis tincidunt. Fusce vitae lacinia metus. Pellentesque habitant morbi. $\mathbf{A}\underline{\xi}=\underline{\beta}$ Vim $\underline{\xi}$ enum nidi $3(P+2)^{2}$ lacina. Id feugain $\mathbf{A}$ nun quis; magno.

%----------------------------------------------------------------------------------------
%	RESULTS 
%----------------------------------------------------------------------------------------

\section*{Results}

Donec faucibus purus at tortor egestas eu fermentum dolor facilisis. Maecenas tempor dui eu neque fringilla rutrum. Mauris \emph{lobortis} nisl accumsan. Aenean vitae risus ante.
%
\begin{wraptable}{l}{12cm} % Left or right alignment is specified in the first bracket, the width of the table is in the second
\begin{tabular}{l l l}
\toprule
\textbf{Treatments} & \textbf{Response 1} & \textbf{Response 2}\\
\midrule
Treatment 1 & 0.0003262 & 0.562 \\
Treatment 2 & 0.0015681 & 0.910 \\
Treatment 3 & 0.0009271 & 0.296 \\
\bottomrule
\end{tabular}
\captionof{table}{\color{Green} Table caption}
\end{wraptable}
%
Phasellus imperdiet, tortor vitae congue bibendum, felis enim sagittis lorem, et volutpat ante orci sagittis mi. Morbi rutrum laoreet semper. Morbi accumsan enim nec tortor consectetur non commodo nisi sollicitudin. Proin sollicitudin. Pellentesque eget orci eros. Fusce ultricies, tellus et pellentesque fringilla, ante massa luctus libero, quis tristique purus urna nec nibh.

Nulla ut porttitor enim. Suspendisse venenatis dui eget eros gravida tempor. Mauris feugiat elit et augue placerat ultrices. Morbi accumsan enim nec tortor consectetur non commodo. Pellentesque condimentum dui. Etiam sagittis purus non tellus tempor volutpat. Donec et dui non massa tristique adipiscing. Quisque vestibulum eros eu. Phasellus imperdiet, tortor vitae congue bibendum, felis enim sagittis lorem, et volutpat ante orci sagittis mi. Morbi rutrum laoreet semper. Morbi accumsan enim nec tortor consectetur non commodo nisi sollicitudin.

\begin{center}\vspace{1cm}
%\includegraphics[width=0.8\linewidth]{placeholder}
\captionof{figure}{\color{Green} Figure caption}
\end{center}\vspace{1cm}

In hac habitasse platea dictumst. Etiam placerat, risus ac.

Adipiscing lectus in magna blandit:

\begin{center}\vspace{1cm}
\begin{tabular}{l l l l}
\toprule
\textbf{Treatments} & \textbf{Response 1} & \textbf{Response 2} \\
\midrule
Treatment 1 & 0.0003262 & 0.562 \\
Treatment 2 & 0.0015681 & 0.910 \\
Treatment 3 & 0.0009271 & 0.296 \\
\bottomrule
\end{tabular}
\captionof{table}{\color{Green} Table caption}
\end{center}\vspace{1cm}

Vivamus sed nibh ac metus tristique tristique a vitae ante. Sed lobortis mi ut arcu fringilla et adipiscing ligula rutrum. Aenean turpis velit, placerat eget tincidunt nec, ornare in nisl. In placerat.

\begin{center}\vspace{1cm}
%\includegraphics[width=0.8\linewidth]{placeholder}
\captionof{figure}{\color{Green} Figure caption}
\end{center}\vspace{1cm}

%----------------------------------------------------------------------------------------
%	CONCLUSIONS
%----------------------------------------------------------------------------------------

\color{SaddleBrown} % SaddleBrown color for the conclusions to make them stand out

\section*{Conclusions}

\begin{itemize}
\item Pellentesque eget orci eros. Fusce ultricies, tellus et pellentesque fringilla, ante massa luctus libero, quis tristique purus urna nec nibh. Phasellus fermentum rutrum elementum. Nam quis justo lectus.
\item Vestibulum sem ante, hendrerit a gravida ac, blandit quis magna.
\item Donec sem metus, facilisis at condimentum eget, vehicula ut massa. Morbi consequat, diam sed convallis tincidunt, arcu nunc.
\item Nunc at convallis urna. isus ante. Pellentesque condimentum dui. Etiam sagittis purus non tellus tempor volutpat. Donec et dui non massa tristique adipiscing.
\end{itemize}

\color{DarkSlateGray} % Set the color back to DarkSlateGray for the rest of the content

%----------------------------------------------------------------------------------------
%	FORTHCOMING RESEARCH
%----------------------------------------------------------------------------------------

\section*{Future Work}

Vivamus molestie, risus tempor vehicula mattis, libero arcu volutpat purus, sed blandit sem nibh eget turpis. Maecenas rutrum dui blandit lorem vulputate gravida. Praesent venenatis mi vel lorem tempor at varius diam sagittis. Nam eu leo id turpis interdum luctus a sed augue. Nam tellus.

%----------------------------------------------------------------------------------------
%	REFERENCES
%----------------------------------------------------------------------------------------

%\nocite{*} % Print all references regardless of whether they were cited in the poster or not
\bibliographystyle{plain} % Plain referencing style
\bibliography{references}

%----------------------------------------------------------------------------------------
%	ACKNOWLEDGEMENTS
%----------------------------------------------------------------------------------------

%\section*{Acknowledgements}
%
%Etiam fermentum, arcu ut gravida fringilla, dolor arcu laoreet justo, ut imperdiet urna arcu a arcu. Donec nec ante a dui tempus consectetur. Cras nisi turpis, dapibus sit amet mattis sed, laoreet.

%----------------------------------------------------------------------------------------

\end{multicols}
\end{document}
