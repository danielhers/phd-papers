%
% File naaclhlt2018.tex

\documentclass[11pt,a4paper]{article}
\usepackage[hyperref]{naaclhlt2018}
\usepackage{times}
\usepackage{latexsym}

\usepackage{url}

%\aclfinalcopy % Uncomment this line for the final submission
%\def\aclpaperid{***} %  Enter the acl Paper ID here

%\setlength\titlebox{5cm}
% You can expand the titlebox if you need extra space
% to show all the authors. Please do not make the titlebox
% smaller than 5cm (the original size); we will check this
% in the camera-ready version and ask you to change it back.

\title{Deep Multitask Learning for Transition-Based Semantic Parsing}

\author{Daniel Hershcovich$^{1,2}$ \\
  \\\And
  Omri Abend$^2$ \\
  $^1$The Edmond and Lily Safra Center for Brain Sciences \\
  $^2$School of Computer Science and Engineering \\
  Hebrew University of Jerusalem \\
  \texttt{\{danielh,oabend,arir\}@cs.huji.ac.il}
  \\\And
  Ari Rappoport$^2$
}

\date{}

\begin{document}
\maketitle
\begin{abstract}
  Semantic representation schemes differ in many ways, but we show
  how they are similar and how this similarity can be exploited to
  improve parsing each of them.
\end{abstract}

\section{Introduction}\label{sec:introduction}

Following increased interest in semantic representation,
recent developments in natural language processing have focused on semantic parsing,
including frame-semantic parsing \cite{gildea2002automatic,swayamdipta2017frame,ringgaard2017sling},
Abstract Meaning Representation parsing \cite{damonte-17,11099},
Semantic Dependency Parsing \cite{P17-1186}, and
Universal Conceptual Cognitive Annotation parsing \cite{hershcovich2017a}, among others.
In parallel, Universal Dependency parsers \cite{dozat2016deep} are improving,
learning syntactic structure in a language-universal way.

While each of these representation schemes has its own set of distinctions it focuses on,
much of the semantic content is shared between many of them \cite{abend2017the}.
Given the success of multitask learning models in various tasks
\cite{luong2015multi,ruder2017overview}
including parsing specifically
\cite{Zhang2016StackpropagationIR,P17-1186,swayamdipta2017frame,guo2016exploiting}
and multilingual parsing \cite{TACL892},
we propose a multitask transition-based semantic parser.


\section{Tasks}

\subsection{Universal Conceptual Cognitive Annotation}

UCCA graphs are labeled, directed acyclic graphs (DAGs),
whose leaves correspond to the tokens of
the text. A node (or {\it unit}) corresponds to a terminal or
to several terminals (not necessarily contiguous) viewed as a
single entity according to semantic or cognitive considerations.
Edges bear a category, indicating the role of the sub-unit in the parent relation.

UCCA is a multi-layered representation, where each layer corresponds
to a ``module'' of semantic distinctions.
UCCA's \textit{foundational layer}, targeted in this paper, covers the predicate-argument
structure evoked by predicates of all grammatical categories
(verbal, nominal, adjectival and others), the inter-relations between them,
and other major linguistic phenomena such as coordination and multi-word expressions.
The layer's basic notion is the \textit{scene},
describing a state, action, movement or some other relation that evolves in time.
Each scene contains one main relation (marked as either a Process or a State),
as well as one or more Participants.
For example, the sentence ``After graduation, John moved to Paris''
contains two scenes, whose main relations are ``graduation'' and ``moved''.
``John'' is a Participant in both scenes, while ``Paris'' only in the latter.
Further categories account for inter-scene relations and the internal structure of
complex arguments and relations (e.g. coordination, multi-word expressions and modification).

One incoming edge for each non-root node is marked as \textit{primary},
and the rest (mostly used for implicit relations and arguments) as \textit{remote} edges,
a distinction made by the annotator.
The primary edges thus form a tree structure, whereas the remote edges enable reentrancy,
forming a DAG.

\subsection{Abstract Meaning Representation}

Abstract Meaning Representation (AMR)
is a semantic representation for natural
language that embeds annotations related
to traditional tasks such as named entity
recognition, semantic role labeling, word
sense disambiguation and co-reference
resolution.

AMRs are rooted and directed
graphs with node and edge labels.
For most sentences in our dataset, the
AMR graph is a directed acyclic graph (DAG),
with a few specific cases where cycles are permitted.
These cases are rare, and for the purpose of
this paper, we consider AMR as DAGs.



\section{Experiments}\label{sec:experiments}

We perform various experiments to evaluate which of the representation schemes benefit each other.
As a baseline, we train the parser separately on each task.

\subsection{Data}

For UCCA, we use the English Wikipedia corpus \cite{abend2013universal},
and in addition, the English, French and German 20K corpus \cite{sulem2015conceptual}.
For AMR, we use LDC2016E25 used in SemEval 2016 \cite{may2017semeval}.
For SDP, we use SemEval 2015 \cite{oepen2015semeval}.
For UD, we use UD v2 used in CoNLL 2017 \cite{zeman2017conll}.
Table~\ref{tab:corpora} shows the size of each corpus.

\begin{table}\label{tab:corpora}
\begin{tabular}{lcc}
Corpus & \# Tokens & \# Sentences \\
\textbf{UCCA} \\
Wiki & 158433 & 5225 \\
20K Leagues & 12339 & 506 \\
20K Leagues fr & 12929 & 547 \\
20K Leagues de & 113524 & 4764 \\
\textbf{UD} \\
English & 254850 & 16622 \\
French & 319253 & 16448 \\
German & 287110 & 15590 \\
\textbf{AMR} \\
LDC2016E25 &  & 39260 \\
bio &  & 6452 \\
\textbf{SDP} \\
SemEval 2015 & 802717 & 35657 \\
\end{tabular}
\caption{Size of each corpus.}
\end{table}

\subsection{Results}


\begin{table}\label{tab:single}
\begin{tabular}{lccc}
Parser \\
\textbf{UCCA} & Labeled Precision & Recall & F1 \\
\textbf{UD} & LAS F1 \\
\textbf{AMR} & Smatch Precision & Recall & F1 \\
\textbf{SDP} & Labeled Precision & Recall & F1 \\
\end{tabular}
\caption{Single-task results.}
\end{table}


\bibliography{references}
\bibliographystyle{acl_natbib}

\end{document}
