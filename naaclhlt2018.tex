%
% File naaclhlt2018.tex

\documentclass[11pt,a4paper]{article}
\usepackage[hyperref]{naaclhlt2018}
\usepackage{times}
\usepackage{latexsym}

\usepackage{url}

%\aclfinalcopy % Uncomment this line for the final submission
%\def\aclpaperid{***} %  Enter the acl Paper ID here

%\setlength\titlebox{5cm}
% You can expand the titlebox if you need extra space
% to show all the authors. Please do not make the titlebox
% smaller than 5cm (the original size); we will check this
% in the camera-ready version and ask you to change it back.

\title{Multitask Transition-based Parsing}

\author{Daniel Hershcovich$^{1,2}$ \\
  \\\And
  Omri Abend$^2$ \\
  $^1$The Edmond and Lily Safra Center for Brain Sciences \\
  $^2$School of Computer Science and Engineering \\
  Hebrew University of Jerusalem \\
  \texttt{\{danielh,oabend,arir\}@cs.huji.ac.il}
  \\\And
  Ari Rappoport$^2$
}

\date{}

\begin{document}
\maketitle
\begin{abstract}
  Semantic representation schemes differ in many ways, but we show
  how they are similar and how this similarity can be exploited to
  improve parsing each of them.
\end{abstract}

\section{Introduction}\label{sec:introduction}

Recent developments in natural language processing have focused on semantic parsing
\cite{hershcovich2017a}
including frame-semantic parsing \cite{gildea2002automatic,swayamdipta2017frame,ringgaard2017sling},
following increased interest in semantic representation \cite{abend2017the}.


\bibliography{references}
\bibliographystyle{acl_natbib}

\end{document}
