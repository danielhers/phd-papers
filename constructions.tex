\documentclass[11pt,a4paper]{article}
\usepackage[hyperref]{acl2017}
\usepackage{times}
\usepackage{url}
\usepackage{latexsym}
\usepackage{times}
\usepackage{url}
\usepackage{amsmath}
\usepackage{breqn}
\usepackage{latexsym}
\usepackage{pgfplotstable}
\usepackage{algorithm2e}
\usepackage{hhline}
\usepackage{multirow}
\usepackage[font=small]{caption}
\usepackage{subcaption}
\usepackage{color}
\usepackage{float}
\usepackage{lipsum,adjustbox}
\usepackage{tikz}
\usepackage{tikz-dependency}
\usetikzlibrary{shapes,fit,calc,er,positioning,intersections,decorations.shapes,mindmap,trees}
\tikzset{decorate sep/.style 2 args={decorate,decoration={shape backgrounds,shape=circle,
      shape size=#1,shape sep=#2}}}
\newcommand{\oa}[1]{\footnote{\color{red} #1}}
\newcommand{\daniel}[1]{\footnote{\color{blue} #1}}
\newcommand{\com}[1]{}
\newcommand{\parser}[1]{TUPA\textsubscript{#1}}
\newcommand{\secref}[1]{Section~\ref{#1}}
\newcommand{\figref}[1]{Figure~\ref{#1}}
\newcommand{\tabref}[1]{Table~\ref{#1}}
\DeclareMathOperator*{\argmin}{argmin}
\DeclareMathOperator*{\argmax}{argmax}
\SetKwRepeat{Do}{do}{while}
\hyphenation{SemEval}
\hyphenation{PARSEVAL}
\hyphenation{DAGParser}
\hyphenation{TurboSemanticParser}
\hyphenation{MaltParser}

%\aclfinalcopy % Uncomment this line for the final submission
%\def\aclpaperid{***} %  Enter the acl Paper ID here

%\setlength\titlebox{5cm}
% You can expand the titlebox if you need extra space
% to show all the authors. Please do not make the titlebox
% smaller than 5cm (the original size); we will check this
% in the camera-ready version and ask you to change it back.

\title{Broad-Coverage Transition-Based UCCA Parsing}

\author{Daniel Hershcovich$^{1,2}$ \And Omri Abend$^2$ \And Ari Rappoport$^2$ \\
  $^1$Edmond and Lily Safra Center for Brain Sciences, Hebrew University of Jerusalem \\
  $^2$School of Computer Science and Engineering, Hebrew University of Jerusalem \\
  \texttt{\{danielh,oabend,arir\}@cs.huji.ac.il}
}

\date{}

\begin{document}
\maketitle


\paragraph{Per-construction evaluation.}
Many linguistic constructions are naturally encoded in UCCA,
although syntactically they may be less easily characterized.

In addition to the overall evaluation on primary and remote edges,
we perform fine-grained evaluation on five linguistic construction types,
which we extract by criteria on UCCA edge labels and POS tags.\footnote{See
Appendix~\ref{appendix:constructions} for a formal listing of the criteria.}
For each construction, we report the F-score resulting from calculating the
mutual edges (see \secref{sec:exp_setup}) only among the edges matching the
criterion in each graph. We evaluate the scores on the UCCA development set.

\begin{table}[ht]
\scalebox{.8}{
\begin{tabular}{l|cccccc}
& \small Light verbs & \small MWE & \small Pred. nouns & \small Pred. adj. & \small Expletives \\
\hline
\#edges & 264 & 835 & 108 & 51 & 10 \\
%\#edges in UCCA test set & 17 & 292 & 721 & 47 & 35 & 22 \\
%\hline
%\multicolumn{7}{l}{\rule{0pt}{2ex} \footnotesize Parser F-scores (\%)} \\
%DAGParser
%& 16 & 73.5 & 33.1 & 5.1 & 4.8 & 26.7 \\
%TurboParser
%& 28.6 & 70.7 & 34.6 & 8.2 & 6.9 & 0 \\
%\textsc{uparse}
%& 32.3 & 73 & 35.9 & 9.1 & 18.2 & 27.6 \\
%MaltParser
%& 41.7 & 74.1 & 35.5 & 5.5 & 6.1 & 54.5 \\
%LSTM Parser
%& 74.3 & 74.1 & 48.8 & 7.3 & 4.4 & 43.8 \\
%\hline
%\parser{Sparse} & 26.7 & 72 & 42.5 & 9.7 & 38 & 38.7 \\
%\parser{Dense} & 0 & 72.1 & 30.9 & 7 & 27.8 & 38.9 \\
%\parser{MLP} \\
%\parser{BiLSTM}
LF (\%) & 79.7 & 54 & 33.1 & 30.8 & 55.6
\end{tabular}
}
\caption{\label{table:constructions_results}
Results of evaluating \parser{BiLSTM} on UCCA edges corresponding to specific linguistic constructions.
Top: number of edges corresponding to each construction in the UCCA development corpus.
Bottom: labeled F-score of \parser{BiLSTM} (in percentages) calculated just on these edges.}
\end{table}

Light verbs are verbs with little semantic content.
They are characterized in UCCA as verbs with the Function label.

Multi-word expressions are annotated as a flat unit, with all terminals as
direct children. Parsing these correctly allows proper handling of
non-analyzable semantic units.

Predicate nouns and adjectives are identified by the respective part of speech
and by a Process or State label (meaning the unit is a Scene).
These represent relations or events not necessarily expressed by verbs,
and thus often ignored by other schemes.

Finally, expletive constructions are simply ``it'' or ``there'' with a Function
label. These are semantically vacuous and should be ignored in co-reference
resolution, for example.


\bibliography{references}
\bibliographystyle{acl_natbib}


\appendix
\section{Rules for Extracting Linguistic Constructions}
\label{appendix:constructions}

\tabref{tab:constructions_rules} lists the rules used for extracting each linguistic
construction, used for fine-grained evaluation (see \tabref{table:constructions_results}).

\begin{table}
\scalebox{.9}{
\begin{tabular}{l|l}
%Aspectual verbs & $e.P=\{\text{VERB}\} \wedge e.\ell=\text{Adverbial}$ \\
Light verbs & $e.P=\{\text{VERB}\} \wedge e.\ell=\text{Function}$ \\
MWE & $|\{e^\prime\in e.C:e^\prime.\ell=\text{Terminal}\}|>1$ \\
Pred. nouns & $\text{ADJ}\not\in e.P \wedge \text{NOUN}\in e.P \wedge {}$
 \\& $e.\ell\in\{\text{Process, State}\} \wedge {}$
 \\ \multicolumn{2}{r}{$\{e^\prime.\ell:e^\prime\in e.C\}
 	\subseteq\{\text{Center, Function, Terminal}\} \wedge {}$}
 \\& $\text{to}\not\in e.T$ \\
Pred. adjectives & $\text{ADJ}\in e.P \wedge \text{NOUN}\not\in e.P \wedge {}$
 \\& $e.\ell\in\{\text{Process, State}\} \wedge {}$
 \\ \multicolumn{2}{r}{$\{e^\prime.\ell:e^\prime\in e.C\}
 	\subseteq\{\text{Center, Function, Terminal}\} \wedge {}$}
 \\& $\text{to}\not\in e.T$ \\
Expletives & $e.T\subseteq\{\text{it, there}\} \wedge e.\ell=\text{Function}$
\end{tabular}
}
\caption{Rules for extracting linguistic constructions.
All rules are on UCCA edges. For an edge $e$ in the graph,
$e.T$ refers to the set of tokens in $e$'s yield,
and $e.P$ to their coarse parts of speech.
$e.C$ refers to the set of outgoing edges from the edge's child,
and $e.\ell$ to the edge's UCCA label.}
\label{tab:constructions_rules}
\end{table}

\end{document}
