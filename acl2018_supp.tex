%
% File acl2018_supp.tex
%

\documentclass[11pt,a4paper]{article}
\usepackage{acl-onecolumn}
\usepackage{times}
\usepackage{url}
\usepackage{latexsym}
\usepackage{amsmath}
\usepackage{breqn}
\usepackage{pgfplotstable}
%\usepackage{algorithm2e}
\usepackage{hhline}
\usepackage{multirow}
\usepackage{multicol}
\usepackage[font=small]{caption}
\usepackage{subcaption}
\usepackage{color}
\usepackage{float}
\usepackage{lipsum,adjustbox}
\usepackage{tikz}
\usepackage{tikz-dependency}
\usepackage{enumitem}
\usepackage{xr}
\externaldocument{tupa_multitask}
\usetikzlibrary{shapes,fit,calc,er,positioning,intersections,decorations.shapes,mindmap,trees}
\tikzset{decorate sep/.style 2 args={decorate,decoration={shape backgrounds,shape=circle,
      shape size=#1,shape sep=#2}}}
\newcommand{\oa}[1]{\footnote{\color{red} #1}}
\newcommand{\daniel}[1]{\footnote{\color{blue} #1}}
\newcommand{\com}[1]{}
\DeclareMathOperator*{\argmin}{argmin}
\DeclareMathOperator*{\argmax}{argmax}
%\SetKwRepeat{Do}{do}{while}

\hyphenation{SemEval}
\hyphenation{PARSEVAL}

\title{Multitask Parsing across Semantic Representation Schemes \\ Supplementary Notes}

\begin{document}
\maketitle

\appendix

\section{Conversion to and from Unified DAG Format}\label{sec:convert}

Although all experiments reported in the main paper on the auxiliary tasks
(AMR, DM and UD) are using unlabeled parsing for these schemes,
our conversion code,
which is attached as supplementary material and is also publicly available at $<$anonymized$>$,
supports full conversion to and from these formats (see \texttt{scheme/conversion} in the code).

Conversion from AMR to the unifid DAG format and back
results in 95\% Smatch $F_1$ \cite{cai2013smatch} when averaged over the
LDC2017T10 test set.
On SDP, the conversion is lossless and results in identical graphs
when converted to UCCA and back.
For UD, and conversion results in 98.5\% LAS $F_1$, due to multi-word tokens,
which are lost as they are not supported in the unified DAG format.

\section{Hyperparameter Settings}\label{sec:hyperparams}
Hyperparameter settings are listed in Table~\ref{tab:hyperparams}.
The middle column shows the hyperparameters used for the single-task architecture,
described in \S\ref{sec:classifier}.
The right column shows the hyperparameters used for the multitask architecture,
described in \S\ref{sec:multitask}.

\textit{Main} refers to parameters specific to the main task---UCCA parsing
(task-specific MLP and BiLSTM, and edge label embedding),
\textit{Aux} to parameters specific to each auxiliary task
(task-specific MLP and BiLSTM, but no edge label embedding since the tasks are unlabeled),
and \textit{Shared} to parameters shared among all tasks
(shared BiLSTM and embeddings).

\textit{External word embeddings} are pre-trained word embeddings (see \S\ref{sec:training}),
and \textit{word embeddings} are randomly initialized.

\begin{table}[h]
\centering
\footnotesize
\begin{tabular}{l|c|ccccc}
\bf Hyperparameter &  \bf Single & \multicolumn{3}{c}{\bf Multitask} \\ 
&& \bf Main & \bf Aux & \bf Shared \\
\hline
external word dim. & 300 &&& 300 \\
word dim. & 200 &&& 200 \\
POS tag dim. & 20 &&& 20 \\
syntactic dep. dim. & 10 &&& 10 \\
named entity dim. & 3 &&& 3 \\
punctuation dim. & 1 &&& 1 \\
action dim. & 3 &&& 3 \\
edge label dim. & 20 & 20 \\
MLP layers & 2 & 2 & 1 \\
MLP dimensions & 50 & 50 & 50 \\
BiLSTM layers & 2 & 2 & 1 & 2 \\
BiLSTM dimensions & 500 & 250 & 50 & 250
\end{tabular}
\caption{Hyperparameter settings.\label{tab:hyperparams}
Middle column: single-task.
Right column: multitask settings for main task (UCCA), auxiliary tasks, and shared parameters.}
\end{table}

\bibliography{references}
\bibliographystyle{acl_natbib}

\end{document}
