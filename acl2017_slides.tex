\documentclass[t]{beamer}
\mode<presentation>

\setbeamertemplate{itemize items}[circle]
\usepackage{palatino}
\usepackage{apalike}
\usepackage{graphicx} % Required for including images
\graphicspath{{figures/}} % Location of the graphics files
\usepackage{booktabs} % Top and bottom rules for table
\usepackage[font=small,labelfont=bf]{caption} % Required for specifying captions to tables and figures
\usepackage{wrapfig} % Allows wrapping text around tables and figures
\usepackage{lipsum,adjustbox}
\usepackage[absolute,overlay]{textpos}
\usepackage{url}
\usepackage{lmodern}
\usepackage{amsmath}
\usepackage{amsfonts}
\usepackage{color}
\usepackage{array}
\usepackage{multirow}
\usepackage{multicol}
\usepackage{tikz}
\usepackage{tikz-dependency}
\usetikzlibrary{arrows.meta,graphs,graphs.standard,graphdrawing,quotes,shapes}
\usegdlibrary{layered}
\captionsetup{labelformat=empty}
\newcommand{\parser}[1]{TUPA\textsubscript{#1}}

\makeatletter
\pgfdeclareshape{vector}{
	  \inheritsavedanchors[from={rectangle}]
	  \inheritbackgroundpath[from={rectangle}]
	  \inheritanchorborder[from={rectangle}]
	  \foreach \x in {center,north east,north west,north,south,south east,south west,east,west}{
	    \inheritanchor[from={rectangle}]{\x}
	  }

    \backgroundpath{
      \pgftransformshift{\pgfpoint{-16pt}{-4pt}}
		  \draw[rounded corners=2pt] (0,0) rectangle (32pt,8pt);
    }

    \beforebackgroundpath{
      \draw[step=8pt,help lines,-] (8pt,.1pt) grid (24pt,7.9pt);
    }
}
\makeatother


\begin{document}


\title[A Transition-Based DAG Parser for UCCA]{A Transition-Based Directed Acyclic Graph Parser for Universal Conceptual Cognitive Annotation}
\author{Daniel Hershcovich, Omri Abend and Ari Rappoport}
\institute[]{\includegraphics[width=.08\pagewidth]{huji_logo.jpg}
\includegraphics[width=.3\pagewidth]{huji_banner.png}}
\date{ACL 2017}

\begin{frame}
\titlepage
\end{frame}


%----------------------------------------------------------------------------------------

\section{Universal Conceptual Cognitive Annotation}

\begin{frame}
\frametitle{Universal Conceptual Cognitive Annotation (UCCA)}
\begin{tikzpicture}[level distance=25mm, sibling distance=23mm, ->,
    every circle node/.append style={fill=black},
    every node/.append style={font=\rmfamily}]
  \node (ROOT) [circle] {}
    child {node (After) {After} edge from parent node[left] {$L$}}
    child {node (graduation) [circle] {}
    {
      child {node {graduation} edge from parent node[left] {$P$}}
    } edge from parent node[left] {$H$} }
    child {node {,} edge from parent node[right] {$U$}}
    child {node (moved) [circle] {}
    {
      child {node (Joe) {Joe} edge from parent node[left] {$A$}}
      child {node {moved} edge from parent node[left] {$P$}}
      child {node [circle] {}
      {
        child {node {to} edge from parent node[left] {$R$}}
        child {node {Paris} edge from parent node[right] {$C$}}
      } edge from parent node[right] {$A$} }
    } edge from parent node[right] {$H$} }
    ;
  \draw[dashed,->] (graduation) to node [auto] {$A$} (Joe);
  \node at (5.6,-.3) {\Large ----- primary edge};
  \node at (5.6,-1.3) {\Large - - - remote edge};
\end{tikzpicture}
\begin{wraptable}{l}{8cm}
  \vspace{-27mm}
  \hspace{14mm} After graduation, Joe moved to Paris
  \begin{adjustbox}{margin=1mm,frame}
  \scalebox{.7}{
  \begin{tabular}{c>{\small\it}l|c>{\small\it}l|c>{\small\it}l}
	  $P$ & process &
	  stack & state &
	  $A$ & participant \\
	  $L$ & linker &
	  $H$ & linked scene &
	  $C$ & center \\
	  $E$ & elaborator &
	  $D$ & adverbial &
	  $R$ & relator \\
	  $N$ & connector &
	  $U$ & punctuation &
	  $F$ & function \\
	  $G$ & ground
  \end{tabular}
  }
  \end{adjustbox}
\end{wraptable}
\end{frame}

\begin{frame}
\frametitle{The UCCA Scheme}
Cross-linguistic semantic scheme \cite{abend2013universal}
\begin{itemize}
\item Rapid and intuitive annotation interface \cite{abend2017uccaapp}
\item Stable in translation \cite{sulem2015conceptual}
\item Useful for MT evaluation \cite{birch2016hume}
\end{itemize}
\vfill
\begin{center}
  \includegraphics[width=\linewidth]{uccaapp.png}
\end{center}
\end{frame}

\begin{frame}
\frametitle{Structural Properties}
\noindent
\begin{enumerate}
\item \color{blue} non-terminal nodes
\item \color{orange} reentrancy
\item \color{red} discontinuity
\end{enumerate}
%(1) {\color{blue} non-terminal nodes}, (2) {\color{orange} reentrancy}, (3) {\color{red} discontinuity}
\centering
%\begin{minipage}{.4\linewidth}{\centering
%  \begin{tikzpicture}[level distance=12mm, sibling distance=16mm, ->,
%      every node/.append style={midway,font=\rmfamily}]
%    \node (ROOT) [fill=blue, circle] {}
%      child {node [fill=blue, circle] {}
%      {
%        child {node {Joe} edge from parent node[left] {\scriptsize $C$}}
%        child {node {and} edge from parent node[left] {\scriptsize $N$}}
%        child {node {Mary} edge from parent node[left] {\scriptsize $C$}}
%      } edge from parent node[left] {\scriptsize $A$} }
%      child {node {went} edge from parent node[left] {\scriptsize $P$}}
%      child {node {home} edge from parent node[left] {\scriptsize $A$}}
%      ;
%  \end{tikzpicture}
%  }
%\end{minipage}
%\hfill
%\begin{minipage}{.4\linewidth}{\centering
%  \begin{tikzpicture}[level distance=12mm, sibling distance=2cm, ->,
%      every node/.append style={midway,font=\rmfamily}]
%    \node (ROOT) [fill=blue, circle] {}
%      child {node {Joe} edge from parent node[left] {\scriptsize $A$}}
%      child {node [fill=blue, circle] {}
%      {
%      	child {node {gave} edge from parent node[left] {\scriptsize $C$}}
%      	child {node (everything) {everything} edge from parent[white]}
%      	child {node {up} edge from parent node[right] {\scriptsize $C$}}
%      } edge from parent node[right] {\scriptsize $P$} }
%      ;
%    \draw[bend right,->, color=red, line width=1pt] (ROOT) to[out=-20, in=180] node [left] {\scriptsize $A$} (everything);
%  \end{tikzpicture}
%  }
%\end{minipage}
%
\vfill
  \begin{tikzpicture}[level distance=12mm, sibling distance=2cm, ->,
      every node/.append style={midway,font=\rmfamily}]
    \node (ROOT) [fill=blue, circle] {}
      child {node (He) {He} edge from parent node[left] {\scriptsize $A$}}
      child {node {decided} edge from parent node[left] {\scriptsize $P$}}
      child {node (totakeaquickshower) [fill=blue, circle] {}
      {
        child {node {to} edge from parent node[left] {\scriptsize $F$}}
        child {node (takeashower) [fill=blue, circle] {}
        {
          child {node {take} edge from parent node[right] {\scriptsize $C$}}
          child {node {a} edge from parent node[right] {\scriptsize $F$}}
          child {node (quick) {quick} edge from parent[white]}
          child {node {shower} edge from parent node[right] {\scriptsize $C$}}
        } edge from parent node[right] {\scriptsize $P$} }
      } edge from parent node[left] {\scriptsize $A$} }
      ;
    \draw[bend left,dashed,->, color=orange, line width=1pt] (takeashower) to node [auto] {\scriptsize $A$} (He);
    \draw[bend left,->, color=red, line width=1pt] (totakeaquickshower) to node [auto] {\scriptsize $D$} (quick);
  \end{tikzpicture}
\vfill
He decided to take a quick shower
\end{frame}



\section{Transition-based UCCA Parsing}

\begin{frame}
\frametitle{Transition-Based Parsing}
\begin{itemize}
 \item Parse text $w_1 \ldots w_n$ to graph $G=(V,E,\ell)$ incrementally.
 \item Classifier determines transition to apply at each step.
 \item Trained by an oracle based on gold-standard graph.
\end{itemize}

\hspace{-1cm}
\scalebox{.92}{
	\begin{tikzpicture}[every node/.append style={font=\rmfamily}]
	\draw[xstep=1cm,ystep=5mm,color=gray] (-0.01,0) grid (1,.5);
	\node[anchor=west] at (0,1) {stack};
	\node[fill=black, circle] at (.5,.25) {};
	\draw[xstep=18mm,ystep=5mm,color=gray] (1.79,0) grid (12.6,.5);
	\node[anchor=west] at (1.9,1) {buffer};
	\node[anchor=west] at (2,.25) {\small After};
	\node[anchor=west] at (3.5,.2) {\small graduation};
	\node[anchor=west] at (5.6,.25) {\small Joe};
	\node[anchor=west] at (7.35,.25) {\small moved};
	\node[anchor=west] at (9.25,.25) {\small to};
	\node[anchor=west] at (10.95,.25) {\small Paris};
	\end{tikzpicture}
}

\vfill
\pause
Transitions defined for UCCA parsing:

\vfill
\{\textsc{Shift, Reduce, Node$_X$, Left-Edge$_X$, Right-Edge$_X$,}\\
\hspace{5mm}\textsc{Left-Remote$_X$, Right-Remote$_X$, Swap, Finish}\}

\vfill
\pause
Supports non-terminal nodes, reentrancy and discontinuity.
\end{frame}

\begin{frame}
\frametitle{Transition-Based Parsing}
\begin{center}
	\begin{tikzpicture}[level distance=15mm, sibling distance=2cm,
    every node/.append style={font=\rmfamily}]
	\draw[xstep=1cm,ystep=5mm,color=gray] (-0.01,0) grid (3,.5);
	\node[anchor=west] at (1,1) {stack};
	\node[fill=black, circle] at (.5,.25) {};
	\node[fill=blue, circle] at (1.5,.25) {};
	\node[anchor=west] at (2,.25) {Joe};
	\draw[xstep=12mm,ystep=5mm,color=gray] (3.59,0) grid (7.2,.5);
	\node[anchor=west] at (4.8,1) {buffer};
	\node[anchor=west] at (3.5,.25) {moved};
	\node[anchor=west] at (5,.25) {to};
	\node[anchor=west] at (6,.25) {Paris};
	\node[anchor=west] at (8,1) {$G$};
	\node[fill=black, circle] at (9,.5) {}
	  child {node  {\small After} edge from parent [->] node[left] {\small L}}
	  child {node [fill=blue, circle] {}
	  {
	    child {node {\small graduation} edge from parent [->] node[right] {\small P}}
	  } edge from parent [->] node[right] {\small H} };
	\node[anchor=west] at (0,-1) {History:};
	\node[anchor=west] at (1.5,-1) {\textsc{Shift},};
	\node[anchor=west] at (1.5,-1.5) {\textsc{Right-Edge\textsubscript L},};
	\node[anchor=west] at (1.5,-2) {\textsc{Reduce},};
	\node[anchor=west] at (1.5,-2.5) {\textsc{Shift},};
	\node[anchor=west] at (1.5,-3) {\textsc{Node\textsubscript P},}; 
	\node[anchor=west] at (1.5,-3.5) {\textsc{Reduce},};
	\node[anchor=west] at (1.5,-4) {\textsc{Shift},};
	\node[anchor=west] at (1.5,-4.5) {\textsc{Right-Edge\textsubscript H},};
	\node[anchor=west] at (1.5,-5) {\textsc{Shift}};
	\end{tikzpicture}
	\captionof{figure}{Intermediate state example.}
\end{center}
\end{frame}

\begin{frame}
\frametitle{Classifiers}
We perform greedy parsing, and experiment with three classifiers:
\begin{flushleft}
	\begin{tabular}{ll}
	\textbf{\parser{Sparse}} & Perceptron with features: \\
	  & words, POS, dependency \& edge label combinations. \\
	\textbf{\parser{MLP}} & 2-layer NN, learned embedding features + \\
	  & external word embeddings. \\
	\textbf{\parser{BiLSTM}} & 2-layer bidirectional LSTM to encode features, \\
	  & 2-layer NN for classification.
	\end{tabular}
\end{flushleft}
\end{frame}



\section{Experiments}

\begin{frame}
\frametitle{Experimental Setup}
\begin{itemize}
 \item Main experiment: UCCA Wikipedia corpus.
 \item Out-of-domain data: English part of English-French parallel corpus,
 	\textit{Twenty Thousand Leagues Under the Sea}.
\end{itemize}

\vfill
\begin{center}
  \includegraphics[width=.5\linewidth]{wikipedia.png}
  \includegraphics[width=.5\linewidth]{squid.jpg}
\end{center}
\end{frame}


\begin{frame}
\frametitle{UCCA Corpora}
\centering
	\begin{tabular}{l|ccc|c}
		& \multicolumn{3}{c|}{Wiki} & 20K \\
		& \small Train & \small Dev & \small Test & Leagues \\
		\hline
		\# passages & 300 & 34 & 33 & 154 \\
		\# sentences & 4268 & 454 & 503 & 506 \\
		\hline
		\# nodes & 298,993 & 33,704 & 35,718 & 29,315 \\
		\% terminal & 42.96 & 43.54 & 42.87 & 42.09 \\
		\% non-term. & 58.33 & 57.60 & 58.35 & 60.01 \\
		\% \textbf{discont.} & \textbf{0.54} & \textbf{0.53} & \textbf{0.44} & \textbf{0.81} \\
		\% \textbf{reentrant} & \textbf{2.38} & \textbf{1.88} & \textbf{2.15} & \textbf{2.03} \\
		\hline
		\# edges & 287,914 & 32,460 & 34,336 & 27,749 \\
		\% primary & 98.25 & 98.75 & 98.74 & 97.73 \\
		\% remote & 1.75 & 1.25 & 1.26 & 2.27 \\
		\hline
		\multicolumn{3}{l}{\footnotesize Average per non-terminal node} \\
		\# children & 1.67 & 1.68 & 1.66 & 1.61 
	\end{tabular}
\captionof{table}{Corpus statistics.}
\end{frame}

\begin{frame}
\frametitle{Evaluation}
%Given gold and predicted graphs, match edges by \textit{yield} and \textit{label}:
%\begin{tikzpicture}[level distance=25mm, sibling distance=23mm, ->,
%    every circle node/.append style={fill=black},
%    every node/.append style={font=\rmfamily}]
%  \node (ROOT) [circle] {}
%    child {node (After) {After} edge from parent node[left] {$L$}}
%    child {node (graduation) [circle] {}
%    {
%      child {node {graduation} edge from parent node[left] {$P$}}
%    } edge from parent node[left] {$H$} }
%    child {node {,} edge from parent node[right] {$U$}}
%    child {node (moved) [circle] {}
%    {
%      child {node (Joe) {Joe} edge from parent node[left] {$A$}}
%      child {node {moved} edge from parent node[left] {$P$}}
%      child {node [circle] {}
%      {
%        child {node {to} edge from parent node[left] {$R$}}
%        child {node {Paris} edge from parent node[right] {$C$}}
%      } edge from parent node[right] {$A$} }
%    } edge from parent node[right] {$H$} }
%    ;
%  \draw[dashed,->] (graduation) to node [auto] {$A$} (Joe);
%  \node at (5.6,-.3) {\Large ----- primary edge};
%  \node at (5.6,-1.3) {\Large - - - remote edge};
%\end{tikzpicture}
\textit{Mutual edges} between predicted graph $G_p=(V_p,E_p,\ell_p)$
and gold graph $G_g=(V_g,E_g,\ell_g)$,
both over terminals $W = \{w_1,\ldots,w_n\}$:
\[
M(G_p,G_g) =
    \left\{(e_1,e_2) \in E_p \times E_g \;|\;
    y(e_1) = y(e_2) \wedge \ell_p(e_1)=\ell_g(e_2)\right\}
\]
The yield $y(e) \subseteq W$ of an edge $e=(u,v)$ in either graph
is the set of terminals in $W$ that are descendants of $v$. \hfill
$\ell$ is the edge label.

\vfill
\pause
Labeled precision, recall and F-score are then defined as:
\[
\text{LP} = \frac{|M(G_p,G_g)|}{|E_p|},\quad
\text{LR} = \frac{|M(G_p,G_g)|}{|E_g|},
\]
\[
\text{LF} = \frac{2 \cdot \text{LP} \cdot \text{LR}}{\text{LP} + \text{LR}}.
\]
Two variants:
one for primary edges, and another for remote edges.
\end{frame}

\begin{frame}
\frametitle{Results}
\begin{center}
	\begin{tabular}{l|ccc|ccc}
	& \multicolumn{3}{c|}{Primary} & \multicolumn{3}{c}{Remote} \\
	& \textbf{LP} & \textbf{LR} & \textbf{LF} & \textbf{LP} & \textbf{LR} & \textbf{LF} \\
	\hline
	Sparse
	& 64.5 & 63.7 & 64.1 & 19.8 & 13.4 & 16 \\
	MLP
	& 65.2 & 64.6 & 64.9 & 23.7 & 13.2 & 16.9 \\
	BiLSTM
	& 74.4 & 72.7 & \textbf{73.5} & 47.4 & 51.6 & \textbf{49.4}
	\end{tabular}
	\captionof{table}{Results on the Wiki test set.}
	
	\vfill
	\pause
	\begin{tabular}{l|ccc|ccc}
	& \multicolumn{3}{c|}{Primary} & \multicolumn{3}{c}{Remote} \\
	& \textbf{LP} & \textbf{LR} & \textbf{LF} & \textbf{LP} & \textbf{LR} & \textbf{LF} \\
	\hline
	Sparse
	& 59.6 & 59.9 & 59.8 & 22.2 & 7.7 & 11.5 \\
	MLP
	& 62.3 & 62.6 & 62.5 & 20.9 & 6.3 & 9.7 \\
	BiLSTM
	& 68.7 & 68.5 & \textbf{68.6} & 38.6 & 18.8 & \textbf{25.3}
	\end{tabular}
	\captionof{table}{Results on the 20K Leagues out-of-domain set.}
\end{center}
\end{frame}

\begin{frame}
\frametitle{Bilexical Graph Approximation}
No existing UCCA parsers $\Rightarrow$ compare to bilexical parsers:
\begin{enumerate}
 \item Convert UCCA to bilexical dependencies
 \item Train bilexical parsers and apply to test sentences
 \item Reconstruct UCCA graphs and compare with gold standard
\end{enumerate}
\vfill

\begin{center}
	\begin{dependency}[theme = simple]
	\begin{deptext}[column sep=.7em,ampersand replacement=\^,font=\rmfamily]
	After \^ graduation \^ , \^ Joe \^ moved \^ to \^ Paris \\
	\end{deptext}
	\depedge{2}{1}{L}
	\depedge{2}{3}{U}
	\depedge[dashed]{2}{4}{A}
	\depedge{5}{4}{A}
	\depedge{2}{5}{H}
	\depedge{7}{6}{R}
	\depedge{5}{7}{A}
	\end{dependency}
	\captionof{figure}{Bilexical DAG approximation.}
\end{center}
\end{frame}

\begin{frame}
\frametitle{Baselines}
Bilexical DAG parsers:
\begin{itemize}
 \item DAGParser \cite{ribeyre-villemontedelaclergerie-seddah:2014:SemEval}:
 transition-based parser
 \item TurboParser \cite{almeida-martins:2015:SemEval}:
 graph-based
\end{itemize}

Tree parsers (all transition-based):
\begin{itemize}
 \item MaltParser \cite{nivre2007maltparser}: bilexical tree parser
 \item LSTM Parser \cite{dyer2015transition}: bilexical tree parser
 \item \textsc{uparse} \cite{maier2015discontinuous}: allows non-terminals, discontinuity
\end{itemize}

\begin{center}
	\begin{dependency}[theme = simple]
	\begin{deptext}[column sep=.7em,ampersand replacement=\^,font=\rmfamily]
	After \^ graduation \^ , \^ Joe \^ moved \^ to \^ Paris \\
	\end{deptext}
	\depedge{2}{1}{L}
	\depedge{2}{3}{U}
	\depedge{5}{4}{A}
	\depedge{2}{5}{H}
	\depedge{7}{6}{R}
	\depedge{5}{7}{A}
	\end{dependency}
	\captionof{figure}{Bilexical tree approximation.}
\end{center}

\vfill
Conversion to trees is just removing remote edges
\end{frame}

\begin{frame}
\frametitle{Results}
\parser{BiLSTM} obtains the highest F-scores in all metrics:
\begin{center}
	\begin{tabular}{l|ccc|ccc}
		& \multicolumn{3}{c|}{Primary} & \multicolumn{3}{c}{Remote} \\
		& \textbf{LP} & \textbf{LR} & \textbf{LF} & \textbf{LP} & \textbf{LR} & \textbf{LF} \\
		\hline
		\parser{Sparse}
		& 64.5 & 63.7 & 64.1 & 19.8 & 13.4 & 16 \\
		\parser{MLP}
		& 65.2 & 64.6 & 64.9 & 23.7 & 13.2 & 16.9 \\
		\parser{BiLSTM}
		& 74.4 & 72.7 & \textbf{73.5} & 47.4 & 51.6 & \textbf{49.4} \\
		\hline
		\footnotesize Bilexical DAG
		& & & \small (91) & & & \small (58.3) \\
		DAGParser
		& 61.8 & 55.8 & 58.6 & 9.5 & 0.5 & 1 \\
		TurboParser
		& 57.7 & 46 & 51.2 & 77.8 & 1.8 & 3.7 \\
		\hline
		\footnotesize Bilexical tree
		& & & \footnotesize (91) & & & \footnotesize -- \\
		MaltParser
		& 62.8 & 57.7 & 60.2 & -- & -- & -- \\
		LSTM Parser
		& 73.2 & 66.9 & 69.9 & -- & -- & -- \\
		\hline
		\footnotesize Tree
		& & & \small (100) & & & \small -- \\
		\textsc{uparse}
		& 60.9 & 61.2 & 61.1 & -- & -- & --
	\end{tabular}
	\captionof{table}{Results on the Wiki test set.}
\end{center}
\end{frame}

\begin{frame}
\frametitle{Results}
Similar on out-of-domain test set:
\begin{center}
	\begin{tabular}{l|ccc|ccc}
		& \multicolumn{3}{c|}{Primary} & \multicolumn{3}{c}{Remote} \\
		& \textbf{LP} & \textbf{LR} & \textbf{LF} & \textbf{LP} & \textbf{LR} & \textbf{LF} \\
		\hline
		\parser{Sparse}
		& 59.6 & 59.9 & 59.8 & 22.2 & 7.7 & 11.5 \\
		\parser{MLP}
		& 62.3 & 62.6 & 62.5 & 20.9 & 6.3 & 9.7 \\
		\parser{BiLSTM}
		& 68.7 & 68.5 & \textbf{68.6} & 38.6 & 18.8 & \textbf{25.3} \\
		\hline
		\footnotesize Bilexical DAG
		& & & \footnotesize (91.3) & & & \footnotesize (43.4) \\
		DAGParser
		& 56.4 & 50.6 & 53.4 & -- & 0 & 0 \\
		TurboParser
		& 50.3 & 37.7 & 43.1 & 100 & 0.4 & 0.8 \\
		\hline
		\footnotesize Bilexical tree
		& & & \footnotesize (91.3) & & & \footnotesize -- \\
		MaltParser
		& 57.8 & 53 & 55.3 & -- & -- & -- \\
		LSTM Parser
		& 66.1 & 61.1 & 63.5 & -- & -- & -- \\
		\hline
		\footnotesize Tree
		& & & \footnotesize (100) & & & \footnotesize -- \\
		\textsc{uparse}
		& 52.7 & 52.8 & 52.8 & -- & -- & --
	\end{tabular}
	\captionof{table}{Results on the 20K Leagues out-of-domain set.}
\end{center}
\end{frame}



\section{Conclusions}

\begin{frame}
\frametitle{Conclusions}
\begin{itemize}
 \item UCCA exhibits formal properties important for semantic representation.
 \item We present the first parser for UCCA and the first to support these properties.
\end{itemize}

\vfill
Future work:
\begin{itemize}
 \item Beam search, training with exploration
 \item More languages (e.g. German)
 \item Different target representations for conversion
 \item NLP applications
\end{itemize}

\vfill
Corpora: \url{http://www.cs.huji.ac.il/~oabend/ucca.html}

Code: \url{https://github.com/danielhers/tupa}

Based on DyNet: \url{https://github.com/clab/dynet}
\end{frame}



\begin{frame}
\vfill
\begin{center}
\LARGE
Thank you
\end{center}
\end{frame}



\begin{frame}[allowframebreaks]
\frametitle{References}
\bibliographystyle{apalike}
\tiny\bibliography{references}
\end{frame}

\end{document}
