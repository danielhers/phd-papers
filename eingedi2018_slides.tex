\documentclass[t,xcolor={svgnames}]{beamer}
%\mode<presentation>

\setbeamertemplate{itemize items}[circle]
\usepackage{pgfpages}
\usepackage{apalike}
\usepackage{graphicx} % Required for including images
\graphicspath{{figures/}} % Location of the graphics files
\usepackage{booktabs} % Top and bottom rules for table
\usepackage[font=small,labelfont=bf]{caption} % Required for specifying captions to tables and figures
\usepackage{wrapfig} % Allows wrapping text around tables and figures
\usepackage{lipsum,adjustbox}
\usepackage[absolute,overlay]{textpos}
\usepackage{url}
\usepackage{lmodern}
\usepackage{amsmath}
\usepackage{amsfonts}
\usepackage{color}
\usepackage{array}
\usepackage{multirow}
\usepackage{multicol}
\usepackage{rotating}
\usepackage{tikz}
\usepackage{tikz-dependency}
\usetikzlibrary{arrows.meta,graphs,graphs.standard,graphdrawing,quotes,shapes}
\usegdlibrary{layered,trees}
\tikzset{
  invisible/.style={opacity=0},
  visible on/.style={alt={#1{}{invisible}}},
  alt/.code args={<#1>#2#3}{%
    \alt<#1>{\pgfkeysalso{#2}}{\pgfkeysalso{#3}} % \pgfkeysalso doesn't change the path
  },
}
\captionsetup{labelformat=empty}

\makeatletter
\pgfdeclareshape{vector}{
	  \inheritsavedanchors[from={rectangle}]
	  \inheritbackgroundpath[from={rectangle}]
	  \inheritanchorborder[from={rectangle}]
	  \foreach \x in {center,north east,north west,north,south,south east,south west,east,west}{
	    \inheritanchor[from={rectangle}]{\x}
	  }

    \backgroundpath{
      \pgftransformshift{\pgfpoint{-16pt}{-4pt}}
		  \draw[rounded corners=2pt] (0,0) rectangle (32pt,8pt);
    }

    \beforebackgroundpath{
      \draw[step=8pt,help lines,-] (8pt,.1pt) grid (24pt,7.9pt);
    }
}
\makeatother


\begin{document}


\begin{frame}
\frametitle{Deep Multitask Learning for Transition-Based DAG Parsing}
\vspace{-5mm}
\begin{center}
    Daniel Hershcovich, Omri Abend and Ari Rappoport
\end{center}

{
  \color{Indigo}
  \begin{center}
    \begin{tikzpicture}[level distance=15mm, -{Latex[length=2mm]},
        every node/.append style={font=\bf\ttfamily},
        every circle node/.append style={fill=Indigo},
      level 1/.style={sibling distance=21mm},
      level 2/.style={sibling distance=13mm}]
      \tikzstyle{word} = [font=\rmfamily,color=black]
      \node (ROOT) [circle] {}
        child {node (After) [word] {After} edge from parent node[above] {L}}
        child {node (hearing) [circle] {}
        {
          child {node [word] {hearing} edge from parent node[left] {P}}
          child {node [word] {this} edge from parent node[left] {A}}
        } edge from parent node[right] {H} }
        child {node [word] {,} edge from parent node[below] {U}}
        child {node (moved) [circle] {}
        {
          child {node (you) [word] {you} edge from parent node[left] {A}}
          child {node [word] {will} edge from parent node[left] {F}}
          child {node [word] {come} edge from parent node[left] {P}}
          child {node [circle] {}
          {
            child {node [word] {to} edge from parent node[left] {R}}
            child {node [circle] {}
            {
              child {node [word] {my} edge from parent node[left] {E}}
              child {node [word] {poster} edge from parent node[right] {C}}
            } edge from parent node[above] {C} }
          } edge from parent node[above] {A} }
        } edge from parent node[above] {H} }
        ;
      \draw[dashed,-{Latex[length=2mm]}] (hearing) to node [above] {A} (you);
    \end{tikzpicture}
  \end{center}
}
\end{frame}


\begin{frame}
{
  \color{Indigo}
  \begin{center}
    \scalebox{.7}{
    \begin{tikzpicture}[level distance=15mm, -{Latex[length=2mm]},
        every node/.append style={font=\bf\ttfamily},
        every circle node/.append style={fill=Indigo},
      level 1/.style={sibling distance=21mm},
      level 2/.style={sibling distance=13mm}]
      \tikzstyle{word} = [font=\rmfamily,color=black]
      \node (ROOT) [circle] {}
        child {node (After) [word] {After} edge from parent node[above] {L}}
        child {node (hearing) [circle] {}
        {
          child {node [word] {hearing} edge from parent node[left] {P}}
          child {node [word] {this} edge from parent node[left] {A}}
        } edge from parent node[right] {H} }
        child {node [word] {,} edge from parent node[below] {U}}
        child {node (moved) [circle] {}
        {
          child {node (you) [word] {you} edge from parent node[left] {A}}
          child {node [word] {will} edge from parent node[left] {F}}
          child {node [word] {come} edge from parent node[left] {P}}
          child {node [circle] {}
          {
            child {node [word] {to} edge from parent node[left] {R}}
            child {node [circle] {}
            {
              child {node [word] {my} edge from parent node[left] {E}}
              child {node [word] {poster} edge from parent node[right] {C}}
            } edge from parent node[above] {C} }
          } edge from parent node[above] {A} }
        } edge from parent node[above] {H} }
        ;
      \draw[dashed,-{Latex[length=2mm]}] (hearing) to node [above] {A} (you);
    \end{tikzpicture}
    }
  \end{center}
}

\vspace{-33mm}

\scalebox{.7}{
  \begin{tikzpicture}[-{Latex[length=2mm]}, color=DarkGreen, level distance=16mm,
      every node/.append style={sloped,anchor=south,auto=false,font=\scriptsize\bf\ttfamily},
      level 1/.style={sibling distance=24mm}]
    \node (ROOT) [draw=DarkGreen,ellipse] {come-01}
      child {node [draw=DarkGreen,ellipse] {after}
      {
        child {node (hear) [draw=DarkGreen,ellipse] {hear-01}
        {
              child {node (thid) [draw=DarkGreen,ellipse] {this} edge from parent node {ARG1} }
        } edge from parent node {op1} }
      } edge from parent node {time} }
      child {node (you) [draw=DarkGreen,ellipse] {you} edge from parent node {ARG1} }
      child {node [draw=DarkGreen,ellipse] {poster}
      {
        child {node (thid) [draw=DarkGreen,ellipse] {i} edge from parent node {poss} }
      } edge from parent node {ARG2} }
      ;
      \draw (hear) to node {ARG0} (you);
  \end{tikzpicture}
  }
\vspace{-15mm}

\begin{flushright}
\scalebox{.7}{
    \begin{dependency}[edge style={-{Latex[length=2mm]}, color=DarkRed},
        text only label, label style={above, color=DarkRed, font=\bf\ttfamily}, font=\small]
    \begin{deptext}[column sep=.8em,ampersand replacement=\^]
    After \^ hearing \^ this \^ , \^ you \^ will \^ come \^ to \^ my \^ poster \\
    \end{deptext}
        \depedge{1}{2}{ARG2}
        \depedge{2}{3}{ARG1}
        \depedge{2}{5}{ARG0}
        \depedge[edge start x offset=-4mm]{7}{5}{ARG1}
        \depedge[edge end x offset=-3mm]{1}{7}{ARG1}
        \deproot{7}{top}
        \depedge[edge start x offset=1mm, edge end x offset=3mm]{7}{10}{ARG2}
        \depedge[edge end x offset=5mm]{8}{7}{ARG1}
        \depedge{8}{10}{ARG2}
        \depedge{9}{10}{poss}
    \end{dependency}
}
\end{flushright}
\end{frame}

\begin{frame}
\begin{itemize}
\item Semantic representation reflects cognitive processing and is important for text understanding
\item Parsing to semantic representations is best done by supervised learning
\item Supervised learning requires enough training data in the right domain
\item But training data for semantic representation is small and in specific domains
\item Different semantic representations focus on different aspects, but much is common
\item Can we bring them together and generalize to improve on each of them?
\end{itemize}
\end{frame}

\end{document}
